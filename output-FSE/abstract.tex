What do you give for free to your competitor when you exhibit a product line? 
This paper addresses this question through several cases in which the discovery of trade secrets of a product line is possible and can lead to severe consequences. 
That is, we show that an outsider can understand the variability realization and gain either confidential business information or even some economical direct advantage.  
For instance, an attacker can identify hidden constraints and bypass the product line to get access to features or copyrighted data.
This paper warns against possible naive modeling, implementation, and testing of variability leading to the existence of 
product lines that jeopardize their trade secrets.
% Malicious attackers
% As a consequence, 
% As product lines jeopardize 
 Our vision is that defensive methods and techniques should be developed to protect specifically variability -- or at least further complicate the task of reverse engineering it.
 % support  
  % We expose and review potential techniques for 
% However succeeding to efficiently model, implement, and test variability is not enough; practitioners should also think about defensive mechanisms for protecting variability.

  
% variability, as a key competing advantage and first-class citizen, should itself vary to complicate the task of an external attacker.
% malicious attackers should not be bale to too easily build mental abstractions, isolate, reason,
% and navigate into the configuration space -- up to the point the reverse engineering and the reengineering of variability is highly facilitated or even immediate.


% too easily build mental abstractions, isolate, reason,
% and navigate into the product line 
% configuration space -- up to the point the reverse engineering and the reengineering of variability is highly facilitated or even immediate. 

% variability  . 
 

% We expose potential techniques for varying variability and call to further investigate the protection perspective onto variability.


% Variability remains undoubtedly a standing goal of any project, making the promise of exposing thousands of configuration options and delivering billions of unique products (or variants) to customers. 
 % However succeeding to efficiently model, implement, and test variability is not enough; practitioners should also think about defensive mechanisms for protecting variability.
 % We present four case studies in which the discovery and understanding of variability is possible and can lead to severe (economical) consequences. In the case studies, we show that a malicious attacker can too easily build mental abstractions, isolate, reason,
% and navigate into the configuration space -- up to the point the reverse engineering and the reengineering of variability is highly facilitated or even immediate. 
% Our vision is that variability, as a key competing advantage and first-class citizen, should itself vary to complicate the task of an external attacker.
% We expose potential techniques for varying variability and call to further investigate the protection perspective onto variability.

% but are almost absent in the product line literature.
 % In other words, variability should be protected otherwise the variability can be too easily reverse engineered. 
  % This paper argues that defensive mechanisms for protecting variability should be further investigated. 
 % alleviating the 
% protection is crucial.  
 
 % and that an external attacker should not be able to reverse engineer it. 


% Variability should be treated as a first-class citizen



